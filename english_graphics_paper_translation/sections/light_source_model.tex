\section{为光源建模}
通常来说,光源都被建模成点光源或者方向光源。
但是,一个实际的光源有有限的尺寸,一个可能不规则的外形,和一个相对于被展示物体的明确的空间位置。
有限尺寸的光源投射出的阴影包括本影和半影。
本影是指没有从光源接收到光线的阴影空间。
半影是指接收到部分但不是全部光线的阴影空间。
所以对于这样的阴影,都有一个深色的中心区域,环之以边界区域,在边界区域里有从有阴影区域到无阴影区域的平滑过渡。
对于一个不规则形状的光源,可以在本影的外围用一条固定宽度的、线性渐变的条带阴影来估计半影。
我们可以期望半影的计算能够显著增强展示阴影的效果。
所以,我们假设用一个点光源,或者一个无限远离的光源(只用方向描述)。\\
阴影边界通过投射一个物体的轮廓到另一个物体上来决定。
使用的投射类型取决于光源的位置。
用来计算阴影的最简单的光源是无限远离的光源。因为阴影边界通过一次正射投影就可以找到。
另一方面,当光源位于物体空间中时,计算阴影边界的难易程度因位置而不同。
如果光源在视场外,阴影边界的计算采用和图片显示一样的透视投影方式。
但是,当光源在视场内时,必须采用不同的方法。
因为传统的透视变换只对有限的视场准确,所以,要么空间必须被划分成以光源为中心的放射状部分,这样就可以用透视变换,要么就要使用更复杂的三维几何方法。\\
透视变换提供了便利和高效。
但是,下面这种方式也总是可以确定物体空间内的阴影边界。
利用光源的位置和物体轮廓来确定表面,然后计算这个表面和其他物体的相交面。