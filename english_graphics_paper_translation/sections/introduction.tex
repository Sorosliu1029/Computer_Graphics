\section{介绍}
现在大多数计算机合成的荫蔽图片有一个主要的缺陷,就是缺少阴影。
尽管当光源和出射点一致时,阴影并不需要。很多早期的实现方式都利用了这一事实。
但是许多对现实感图片有严肃要求的应用(像太空飞船的入轨,飞机的着陆模拟)都要求有阳光照射的图片。
现在已经可以生成包含阴影的很真实的场景图了,但是这些图片的成败依赖于有一个漫射光源的假设,像多云天气时的阳光那种。\\
有些情况下,阴影是很重要的。
一个投射阴影可以让设备的一块重要组件在实际情况下事实上不可见,即使它可以在没有阴影的模拟环境下清晰地显示。
计算机图形学应用到建筑坐落问题,环境影响调查方面时,要求阴影的计算,用来评估空调的需求,或者太阳能的可用性。
更重要的是,阴影提供了有价值的位置信息。一个物体投射到另一个物体上的阴影可以表明空间关系,不然的话,就是模糊的空间关系。
此外,阴影造成了一个有趣的问题。它们应该得到比现在更多的关注。\\
Appel\cite{3}和之后的Bouknight、Kelley\cite{5}展示了对于阴影问题的解决方案,之后在这篇论文中会用对阴影算法分类的方式来讨论这些解决方案。
现在有三类解决方案是可以确认的(也许未来会有还未发现的类别)。
Appel, Bouknight和Kelley提出了一类的解决方案和算法,而另两类提出了但还没有实现。\\
第一类算法,由Appel, Bouknight和Kelley论证,在图片从光栅扫描器生成时检测阴影边界。
通过投影潜在的阴影多边形的边到正在扫描的表面,可以找到投射阴影的边。
由此形成的阴影边之后投影到物体平面。
当穿越一条阴影边时,扫描片段的颜色也会恰当地变换。\\
第二类算法包括两遍隐藏表面算法,或者也许是两种不同算法各只进行一次。
第一遍区分有阴影表面和无阴影表面,然后从和光源一致的角度决定隐藏表面,划分部分有阴影的表面。
接着,有阴影表面的颜色会被修改。
第二遍从观察者的角度来操作这些增强了的数据。\\
第三类阴影算法包括计算一个表面,这个表面包围被物体阴影(它的本影)遮盖的空间体积。
这个本影表面之后被添加到数据中,并且被当作一个不可见的表面。当这个表面被穿透时,会产生穿进或穿出物体阴影的变化。\\
一个对于这三类方案更加完整的解释会随着每个类别建议的实现方式一起给出。
在此之前,会有对光源建模的回顾。
在此之后,是尝试性的对于实践过程中实现这三种途径的难度的比较。