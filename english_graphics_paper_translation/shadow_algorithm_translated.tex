\documentclass[a4paper, twocolumn]{article}
\usepackage{CJKutf8}
\usepackage[sc]{mathpazo} % Use the Palatino font
\usepackage[T1]{fontenc} % Use 8-bit encoding that has 256 glyphs
\linespread{1.2} % Line spacing - Palatino needs more space between lines
\usepackage{microtype} % Slightly tweak font spacing for aesthetics

\usepackage[english]{babel} % Language hyphenation and typographical rules

\usepackage[hmarginratio=1:1,top=32mm,left=20mm,right=20mm,columnsep=20pt]{geometry} % Document margins
\usepackage[hang, small,labelfont=bf,up,textfont=it,up]{caption} % Custom captions under/above floats in tables or figures
%\usepackage{booktabs} % Horizontal rules in tables

\usepackage{lettrine} % The lettrine is the first enlarged letter at the beginning of the text

\usepackage{enumitem} % Customized lists
\setlist[itemize]{noitemsep} % Make itemize lists more compact

\usepackage{abstract} % Allows abstract customization
\renewcommand{\abstractnamefont}{\normalfont\itshape\bfseries} % Set the "Abstract" text to bold
\renewcommand{\abstracttextfont}{\normalfont} % Set the abstract itself to small italic text

\usepackage{titlesec} % Allows customization of titles
\renewcommand\thesection{\Roman{section}} % Roman numerals for the sections
\renewcommand\thesubsection{\roman{subsection}} % roman numerals for subsections
\titleformat{\section}[block]{\large\scshape\centering}{\thesection.}{1em}{} % Change the look of the section titles
\titleformat{\subsection}[block]{\large}{\thesubsection.}{1em}{} % Change the look of the section titles

\usepackage{fancyhdr} % Headers and footers
\pagestyle{fancy} % All pages have headers and footers
\fancyhead{} % Blank out the default header
\fancyfoot{} % Blank out the default footer
\fancyhead[C]{
	\begin{CJK}{UTF8}{gbsn}
	《计算机图形学》课程 论文翻译
	\end{CJK}}
\fancyfoot[RO,LE]{\thepage} % Custom footer text
\usepackage{titling} % Customizing the title section

\usepackage{hyperref} % For hyperlinks in the PDF
\hypersetup{hidelinks}

\usepackage{graphicx}
\renewcommand{\normalsize}{\fontsize{10.5pt}{\baselineskip}\selectfont}
%----------------------------------------------------------------------------------------
%	TITLE SECTION
%----------------------------------------------------------------------------------------

\setlength{\droptitle}{-4\baselineskip} % Move the title up

\pretitle{\begin{center}\Huge\bfseries} % Article title formatting
	\posttitle{\end{center}} % Article title closing formatting
\title{计算机图形学中的阴影算法} % Article title
\author{%
	\textsc{Franklin C. Crow} \\[1ex] % Your name
	\normalsize University of Texas at Austin \\ % Your institution
	\normalsize Austin, Texas \\ % Your address 
	\normalsize 原论文链接:\href{http://dl.acm.org/citation.cfm?id=563901}{http://dl.acm.org/citation.cfm?id=563901} \\
	\normalsize 译者:刘阳
}
\date{} % Leave empty to omit a date

%----------------------------------------------------------------------------------------

\begin{document}
\begin{CJK}{UTF8}{gbsn}
	% Print the title
	\maketitle
	
	%----------------------------------------------------------------------------------------
%		ARTICLE CONTENTS
	%----------------------------------------------------------------------------------------
	\section{摘要}
在计算机合成的图片中,提倡用阴影来提升感知性,增强现实感。
一种对于阴影算法的分类方式描述了三种途径:扫描时进行阴影计算;在移除隐藏表面前先将物体表面分为有阴影区域和无阴影区域;在物体数据中加入阴影量。
这些类别关联到已经存在的阴影算法,并且各个类别的实现都有概述。
一次对于这三类途径的简要比较表明最后一种途径有最吸引人的特征。
	\section{介绍}
现在大多数计算机合成的荫蔽图片有一个主要的缺陷,就是缺少阴影。
尽管当光源和出射点一致时,阴影并不需要。很多早期的实现方式都利用了这一事实。
但是许多对现实感图片有严肃要求的应用(像太空飞船的入轨,飞机的着陆模拟)都要求有阳光照射的图片。
现在已经可以生成包含阴影的很真实的场景图了,但是这些图片的成败依赖于有一个漫射光源的假设,像多云天气时的阳光那种。\\
有些情况下,阴影是很重要的。
一个投射阴影可以让设备的一块重要组件在实际情况下事实上不可见,即使它可以在没有阴影的模拟环境下清晰地显示。
计算机图形学应用到建筑坐落问题,环境影响调查方面时,要求阴影的计算,用来评估空调的需求,或者太阳能的可用性。
更重要的是,阴影提供了有价值的位置信息。一个物体投射到另一个物体上的阴影可以表明空间关系,不然的话,就是模糊的空间关系。
此外,阴影造成了一个有趣的问题。它们应该得到比现在更多的关注。\\
Appel\cite{3}和之后的Bouknight、Kelley\cite{5}展示了对于阴影问题的解决方案,之后在这篇论文中会用对阴影算法分类的方式来讨论这些解决方案。
现在有三类解决方案是可以确认的(也许未来会有还未发现的类别)。
Appel, Bouknight和Kelley提出了一类的解决方案和算法,而另两类提出了但还没有实现。\\
第一类算法,由Appel, Bouknight和Kelley论证,在图片从光栅扫描器生成时检测阴影边界。
通过投影潜在的阴影多边形的边到正在扫描的表面,可以找到投射阴影的边。
由此形成的阴影边之后投影到物体平面。
当穿越一条阴影边时,扫描片段的颜色也会恰当地变换。\\
第二类算法包括两遍隐藏表面算法,或者也许是两种不同算法各只进行一次。
第一遍区分有阴影表面和无阴影表面,然后从和光源一致的角度决定隐藏表面,划分部分有阴影的表面。
接着,有阴影表面的颜色会被修改。
第二遍从观察者的角度来操作这些增强了的数据。\\
第三类阴影算法包括计算一个表面,这个表面包围被物体阴影(它的本影)遮盖的空间体积。
这个本影表面之后被添加到数据中,并且被当作一个不可见的表面。当这个表面被穿透时,会产生穿进或穿出物体阴影的变化。\\
一个对于这三类方案更加完整的解释会随着每个类别建议的实现方式一起给出。
在此之前,会有对光源建模的回顾。
在此之后,是尝试性的对于实践过程中实现这三种途径的难度的比较。
	\section{为光源建模}
通常来说,光源都被建模成点光源或者方向光源。
但是,一个实际的光源有有限的尺寸,一个可能不规则的外形,和一个相对于被展示物体的明确的空间位置。
有限尺寸的光源投射出的阴影包括本影和半影。
本影是指没有从光源接收到光线的阴影空间。
半影是指接收到部分但不是全部光线的阴影空间。
所以对于这样的阴影,都有一个深色的中心区域,环之以边界区域,在边界区域里有从有阴影区域到无阴影区域的平滑过渡。
对于一个不规则形状的光源,可以在本影的外围用一条固定宽度的、线性渐变的条带阴影来估计半影。
我们可以期望半影的计算能够显著增强展示阴影的效果。
所以,我们假设用一个点光源,或者一个无限远离的光源(只用方向描述)。\\
阴影边界通过投射一个物体的轮廓到另一个物体上来决定。
使用的投射类型取决于光源的位置。
用来计算阴影的最简单的光源是无限远离的光源。因为阴影边界通过一次正射投影就可以找到。
另一方面,当光源位于物体空间中时,计算阴影边界的难易程度因位置而不同。
如果光源在视场外,阴影边界的计算采用和图片显示一样的透视投影方式。
但是,当光源在视场内时,必须采用不同的方法。
因为传统的透视变换只对有限的视场准确,所以,要么空间必须被划分成以光源为中心的放射状部分,这样就可以用透视变换,要么就要使用更复杂的三维几何方法。\\
透视变换提供了便利和高效。
但是,下面这种方式也总是可以确定物体空间内的阴影边界。
利用光源的位置和物体轮廓来确定表面,然后计算这个表面和其他物体的相交面。
	\section{第一类:扫描时进行阴影计算}
	\section{第二类:两遍的方式}
	\section{第三类:投射阴影多边形}
	\section{三类方法间的比较}
关于在使用上述方法表示阴影时的附加困难,我们可以做个比较。
我们使用三个比较标准:
需要的额外数据存储空间,
需要的额外计算量,
所需的额外软件的实现难易。
在判定的时候,我们假设用扫描隐藏表面算法。\\
初看一眼,只有第二类和第三类算法看起来需要额外的数据存储。
两遍的方法要求表面沿着阴影边界分开,或者至少要在数据中包含阴影边界。
阴影多边形的方法要求存储可能的大量阴影多边形。
但是,这两类算法都不要求整个场景描述,用来进行隐藏表面计算。
背面表面和视场之外的数据可以被丢弃。
另一方面,第一类算法要求所有物体一直以原始数据的方式呈现,这样投影阴影边界才能在扫描期间计算出来。
所以,留给隐藏表面算法的后续步骤所需的临时空间就会严重缩减。
一定可以得出的结论是:两遍的方式需要最少的额外存储空间,阴影多边形的方式需要稍多一些的空间,而扫描时计算阴影的方式需要最多的空间。\\
假设仅使用外围来计算阴影边界总是更高效的,那么投射阴影多边形的方式似乎产生的必要计算的增加的是最少的。
一旦找到了外围边,那么阴影多边形的定义是很直观的。
通过利用阴影多边形的特殊性质,在扫描期间要做的额外计算量是最少的。
更进一步,另外两种方式要求方法遵循更低要求的增长律。
在扫描时计算阴影的方式要求额外的计算量,这是为了决定哪些表面可能投射阴影到其他表面上。并且通过操作物体空间数据,要求计算分割开有阴影区域和无阴影区域的线段。
Bouknight和Kelley概略地汇报说每次单个场景的阴影计算会扩大两倍的计算量。
反之,两遍的方式要求对于隐藏表面算法的额外解决方案。
但是,因为只要考虑外围边,第一遍可以简化。\\
对于投射阴影多边形的方式而言,所需的额外软件的复杂度同样是最低的。
第一类和第二类算法都要求完全新的软件。
不过,可以说,一旦一个合适的,针对于两遍的方式的隐藏表面算法出现了,那么第一遍所需的软件仅仅是第二遍所需软件的子集,所以并不需要额外的软件。\\
给定一个有可用扫描隐藏表面算法的情景,看起来阴影多边形的方式提供了最佳解决方案。
不过,从头开始的话,并没有一个清晰的最佳选择。
当然,通过实现任何一类的算法,也可以学到很多东西。
	\section{鸣谢}
这里表述的大部分想法源自于和Utah大学同事的交流。
尤其是Ivan Sutherland建议我关注投射阴影多边形的概念,还提供了很多对这篇论文初稿的有用的建议。
	
	%----------------------------------------------------------------------------------------
	%	REFERENCE LIST
	%----------------------------------------------------------------------------------------
	\renewcommand{\refname}{参考文献}
	\begin{thebibliography}{99} % Bibliography - this is intentionally simple in this template
		
		\bibitem{1}
		Appel, A.,
		\newblock {\em The Notion of Quantitative Invisibility and the Machine Rendering of Solids, } 
		\newblock Proceedings ACM 1967 National Conference.
		
		\bibitem{2}
		Appel, A.,
		\newblock {\em Some Techniques for Shading Machine Renderings of Solids, } 
		\newblock 1968 SJCC, AFIPS Vol. 32.
		
		\bibitem{3}
		Appel, A.,
		\newblock {\em On Calculating the Illusion of Reality, } 
		\newblock IFIP 1968.
		
		\bibitem{4}
		Bouknight, W. J.,
		\newblock {\em A Procedure for the Generation of 3-D Half-Toned Computer Graphics Presentations,} 
		\newblock CACM, Vol. 13, no. 6, Sept. 1970.
		
		\bibitem{5}
		Bouknight, W. J. and Kelley, K.,
		\newblock {\em An Algorithm for Producing Half-Tone Computer Graphics Presentations with Shadows and Moveable Light Sources,} 
		\newblock 1970 SJCC, AFIPS Vol. 36.
		
		\bibitem{6}
		Bui Tuong Phong and Crow, F. C.,
		\newblock {\em Improved Rendition of Polygonal Models of Curved Surfaces,} 
		\newblock Proc. of the 2nd USA-Japan Computer Conf., 1975.
		
		\bibitem{7}
		Clark, J. H.,
		\newblock {\em Hierarchical Geometric Models for Visible Surface Algorithms,} 
		\newblock CACM, Vol. 19 no. 10, Oct. 1976.
		
		\bibitem{8}
		Crow, F. C.,
		\newblock {\em The Aliasing Problem in Computer- Synthesized Shaded Images,} 
		\newblock Dept of Computer Science University of Utah, UTEC-CSc-76-015, March 1976. (abridged version to appear in CACM)
		
		\bibitem{9}
		Newell, M. G., Newell, R. G. and Sancha, T. L.
		\newblock {\em A Solution to the Hidden-Surface Problem,} 
		\newblock Proceedings of the 1972 ACM National Conference.
		
		\bibitem{10}
		Newell, M. G.,
		\newblock {\em The Utilization of Procedural Models in Digital Image Synthesis,} 
		\newblock Department of Computer Science, University of Utah, UTEC-CSc-76-218, Summer 1975.
		
		\bibitem{11}
		Sutherland, I. E.,
		\newblock {\em Polygon Sorting by Subdivision: A Solution to the Hidden-Surface Problem,} 
		\newblock Unpublished, 1973.
		
		\bibitem{12}
		Sutherland, I. E., Sproull, R. F. and Schu- maker, R. G.,
		\newblock {\em A Characterization of Ten Hidden- Surface Algorithms,} 
		\newblock Computing Surveys, Vol. 6, No. 1, March 1974.
		
\end{thebibliography}
	
	%----------------------------------------------------------------------------------------
\end{CJK}	
\end{document}
