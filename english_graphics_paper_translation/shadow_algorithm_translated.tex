\documentclass[a4paper, twocolumn]{article}
\usepackage{CJKutf8}
\usepackage[sc]{mathpazo} % Use the Palatino font
\usepackage[T1]{fontenc} % Use 8-bit encoding that has 256 glyphs
\linespread{1.2} % Line spacing - Palatino needs more space between lines
\usepackage{microtype} % Slightly tweak font spacing for aesthetics

\usepackage[english]{babel} % Language hyphenation and typographical rules

\usepackage[hmarginratio=1:1,top=32mm,left=25mm,right=25mm,columnsep=20pt]{geometry} % Document margins
\usepackage[hang, small,labelfont=bf,up,textfont=it,up]{caption} % Custom captions under/above floats in tables or figures
%\usepackage{booktabs} % Horizontal rules in tables

\usepackage{lettrine} % The lettrine is the first enlarged letter at the beginning of the text

\usepackage{enumitem} % Customized lists
\setlist[itemize]{noitemsep} % Make itemize lists more compact

\usepackage{abstract} % Allows abstract customization
\renewcommand{\abstractnamefont}{\normalfont\itshape\bfseries} % Set the "Abstract" text to bold
\renewcommand{\abstracttextfont}{\normalfont} % Set the abstract itself to small italic text

\usepackage{titlesec} % Allows customization of titles
\renewcommand\thesection{\Roman{section}} % Roman numerals for the sections
\renewcommand\thesubsection{\roman{subsection}} % roman numerals for subsections
\titleformat{\section}[block]{\large\scshape\centering}{\thesection.}{1em}{} % Change the look of the section titles
\titleformat{\subsection}[block]{\large}{\thesubsection.}{1em}{} % Change the look of the section titles

\usepackage{fancyhdr} % Headers and footers
\pagestyle{fancy} % All pages have headers and footers
\fancyhead{} % Blank out the default header
\fancyfoot{} % Blank out the default footer
\fancyhead[C]{
	\begin{CJK}{UTF8}{gbsn}
	《计算机图形学》课程 论文翻译
	\end{CJK}}
\fancyfoot[RO,LE]{\thepage} % Custom footer text
\usepackage{titling} % Customizing the title section

\usepackage{hyperref} % For hyperlinks in the PDF
\hypersetup{hidelinks}

\usepackage{graphicx}
\renewcommand{\normalsize}{\fontsize{10.5pt}{\baselineskip}\selectfont}
%----------------------------------------------------------------------------------------
%	TITLE SECTION
%----------------------------------------------------------------------------------------

\setlength{\droptitle}{-4\baselineskip} % Move the title up

\pretitle{\begin{center}\Huge\bfseries} % Article title formatting
	\posttitle{\end{center}} % Article title closing formatting
\title{计算机图形学中的阴影算法} % Article title
\author{%
	\textsc{Franklin C. Crow} \\[1ex] % Your name
	\normalsize University of Texas at Austin \\ % Your institution
	\normalsize Austin, Texas \\ % Your address 
	\normalsize 原论文链接:\href{http://dl.acm.org/citation.cfm?id=563901}{http://dl.acm.org/citation.cfm?id=563901} \\
	\normalsize 译者:刘阳
}
\date{} % Leave empty to omit a date

%----------------------------------------------------------------------------------------

\begin{document}
\begin{CJK}{UTF8}{gbsn}
	% Print the title
	\maketitle
	
	%----------------------------------------------------------------------------------------
%		ARTICLE CONTENTS
	%----------------------------------------------------------------------------------------
	
	\section{摘要}
	
	\section{介绍}
	
	\section{为光源建模}
	
	\section{第一类:扫描时进行阴影计算}
	
	\section{第二类:两遍的方式}
	
	\section{第三类:投射阴影多边形}
	
	\section{三类方法间的比较}
	
	\section{鸣谢}
	
	%----------------------------------------------------------------------------------------
	%	REFERENCE LIST
	%----------------------------------------------------------------------------------------
	\renewcommand{\refname}{参考文献}
	\begin{thebibliography}{99} % Bibliography - this is intentionally simple in this template
		
		\bibitem{1}
		Appel, A.,
		\newblock {\em The Notion of Quantitative Invisibility and the Machine Rendering of Solids, } 
		\newblock Proceedings ACM 1967 National Conference.
		
		\bibitem{2}
		Appel, A.,
		\newblock {\em Some Techniques for Shading Machine Renderings of Solids, } 
		\newblock 1968 SJCC, AFIPS Vol. 32.
		
		\bibitem{3}
		Appel, A.,
		\newblock {\em On Calculating the Illusion of Reality, } 
		\newblock IFIP 1968.
		
		\bibitem{4}
		Bouknight, W. J.,
		\newblock {\em A Procedure for the Generation of 3-D Half-Toned Computer Graphics Presentations,} 
		\newblock CACM, Vol. 13, no. 6, Sept. 1970.
		
		\bibitem{5}
		Bouknight, W. J. and Kelley, K.,
		\newblock {\em An Algorithm for Producing Half-Tone Computer Graphics Presentations with Shadows and Moveable Light Sources,} 
		\newblock 1970 SJCC, AFIPS Vol. 36.
		
		\bibitem{6}
		Bui Tuong Phong and Crow, F. C.,
		\newblock {\em Improved Rendition of Polygonal Models of Curved Surfaces,} 
		\newblock Proc. of the 2nd USA-Japan Computer Conf., 1975.
		
		\bibitem{7}
		Clark, J. H.,
		\newblock {\em Hierarchical Geometric Models for Visible Surface Algorithms,} 
		\newblock CACM, Vol. 19 no. 10, Oct. 1976.
		
		\bibitem{8}
		Crow, F. C.,
		\newblock {\em The Aliasing Problem in Computer- Synthesized Shaded Images,} 
		\newblock Dept of Computer Science University of Utah, UTEC-CSc-76-015, March 1976. (abridged version to appear in CACM)
		
		\bibitem{9}
		Newell, M. G., Newell, R. G. and Sancha, T. L.
		\newblock {\em A Solution to the Hidden-Surface Problem,} 
		\newblock Proceedings of the 1972 ACM National Conference.
		
		\bibitem{10}
		Newell, M. G.,
		\newblock {\em The Utilization of Procedural Models in Digital Image Synthesis,} 
		\newblock Department of Computer Science, University of Utah, UTEC-CSc-76-218, Summer 1975.
		
		\bibitem{11}
		Sutherland, I. E.,
		\newblock {\em Polygon Sorting by Subdivision: A Solution to the Hidden-Surface Problem,} 
		\newblock Unpublished, 1973.
		
		\bibitem{12}
		Sutherland, I. E., Sproull, R. F. and Schu- maker, R. G.,
		\newblock {\em A Characterization of Ten Hidden- Surface Algorithms,} 
		\newblock Computing Surveys, Vol. 6, No. 1, March 1974.
		
	\end{thebibliography}
	
	%----------------------------------------------------------------------------------------
\end{CJK}	
\end{document}
